%% start of file `template.tex'.
%% Copyright 2006-2013 Xavier Danaux (xdanaux@gmail.com).
%
% This work may be distributed and/or modified under the
% conditions of the LaTeX Project Public License version 1.3c,
% available at http://www.latex-project.org/lppl/.

\documentclass[11pt,letterpaper,roman,colorlinks=true]{moderncv}

% use linux libertine as the serif font / linux biolinum as the sans serif font
\usepackage[tt=false, type1=true]{libertine}
\usepackage[varqu]{zi4}
\usepackage[libertine]{newtxmath}
\usepackage{makecell} % line breaks inside table cells


\usepackage{xcolor}
\definecolor{accent}{HTML}{007985}
\definecolor{color0}{rgb}{0,0,0}% black
\colorlet{color1}{accent}% accent
\definecolor{color2}{rgb}{0.45,0.45,0.45}% dark grey

\usepackage{enumitem}
\usepackage{etaremune} % reversed-order lists
% print brackets around list index
% https://tex.stackexchange.com/a/428870
\renewcommand*{\labelenumi}{[\theenumi]}

% for some reason this has to go inside document
% https://tex.stackexchange.com/a/572408
% \usepackage[colorlinks]{hyperref}
\AtBeginDocument{%
\hypersetup{
  urlcolor={color2},
  linkcolor={color2},
}
}

%=== bibliography stuff
\usepackage[
            datamodel=acmdatamodel, % acm stuff
            sorting=none,        % show refs in text order
            style=acmnumeric,
            % bibstyle=numeric,    % number refs
            giveninits=false,    % print full author names
            maxbibnames=99,      % list all author names in references
            maxcitenames=99      % list all author names in citations (since our styled references are secretly citations)
           ]{biblatex}

% \DeclareNameAlias{default}{given-family}
% \DeclareNameAlias{sortname}{given-family}
\bibliography{publications}                        % 'publications' is the name of a BibTeX file

\newcommand{\me}[1]{\textbf{#1}}
\newcommand{\HonorableMention}[1]{\\{\href{#1}{\color{accent}[Best Paper, Honorable Mention]}}}

% highlight name in references based on "highlight" annotation from bib file
% https://tex.stackexchange.com/a/304968
\renewcommand*{\mkbibnamegiven}[1]{%
  \ifitemannotation{highlight}
    {\me{#1}}
    {#1}}

\renewcommand*{\mkbibnamefamily}[1]{%
  \ifitemannotation{highlight}
    {\me{#1}}
    {#1}}

\newcommand{\nameuse}[1]{%
  \def\do##1{\settoggle{blx@use##1}{#1}}%
  \dolistcsloop{blx@datamodel@names}
  }

%=== Custom cite command with a newline for authors and journal
% https://tex.stackexchange.com/a/401936
\DeclareCiteCommand{\rawcvcite}
  {\usebibmacro{prenote}}
  {\usebibmacro{author}
   \newline\nopunct\newblock\usebibmacro{maintitle+title}
   \newline\nopunct\newblock
   \usebibmacro{journal+issuetitle}
   \usebibmacro{year}
   \usebibmacro{doi+eprint+url}
   \usebibmacro{addendum+pubstate}
  }
  {\multicitedelim}
  {\usebibmacro{postnote}}

% use minipage to vertically center enumeration label next to citation
\newcommand{\cvcite}[1]{
    \begin{minipage}{\linewidth}
    \rawcvcite{#1}
    \end{minipage}
}

% format the titles in italics
\DeclareFieldFormat
  [article,book,inbook,incollection,inproceedings,patent,thesis,unpublished]
  {title}{{\textit{#1}}}
\DeclareFieldFormat
  [article,book,inbook,incollection,inproceedings,patent,thesis,unpublished]
  {journaltitle}{{{#1}}}

% use title case
\DeclareFieldFormat{titlecase}{#1}

%=== number refs in reverse order https://tex.stackexchange.com/a/22770
% Count total number of entries in each refsection
\AtDataInput{%
  \csnumgdef{entrycount:\therefsection}{%
    \csuse{entrycount:\therefsection}+1}}

% Print the labelnumber as the total number of entries in the
% current refsection, minus the actual labelnumber, plus one
\DeclareFieldFormat{labelnumber}{\mkbibdesc{#1}}    
\newrobustcmd*{\mkbibdesc}[1]{%
  \number\numexpr\csuse{entrycount:\therefsection}+1-#1\relax}

%==== moderncv themes
\moderncvstyle{classic}
%\nopagenumbers{}

% character encoding
\usepackage[utf8]{inputenc}

% adjust the page margins
\usepackage[margin=.75in]{geometry}

% if you want to change the width of the column with the dates
%\setlength{\hintscolumnwidth}{3cm}

% personal data
\name{Mark}{Gillespie\vspace{-.4\baselineskip}}
\title{Curriculum Vitae}

%\address{street and number}{postcode city}{country}
\email{mark.gillespie@inria.fr}
\homepage{www.markjgillespie.com} 
\extrainfo{\faGraduationCap{} \href{https://scholar.google.com/citations?user=o_OqGM0AAAAJ}{google scholar}}
\social[orcid]{0009-0000-5645-9636}
% \phone[mobile]{+1~(908)~477~1822}
\social[github]{MarkGillespie}
% '64pt' is the height the picture must be resized to, 0.4pt is the thickness of the frame around it (put it to 0pt for no frame)
%\photo[64pt][0.4pt]{card_image}  
%\quote{Some quote}

%----------------------------------------------------------------------------------
%            content
%----------------------------------------------------------------------------------
\begin{document}
\newgeometry{margin=.75in, top=.6in}

\makecvtitle
\vspace{-3\baselineskip}

\section{Academic Appointments}
\cventry{{Sept. 2024}--\\present}{Postdoctoral Researcher}{École Polytechnique}{Palaiseau, France}{}{}

\section{Education}
\cventry{2018--2024}{PhD Computer Science}{Carnegie Mellon University}{}{}{Advisor: Keenan Crane}% . Topics: geometry processing, computer graphics}
\cventry{2014--2018}{B.S. Computer Science \& B.S. Mathematics}{California Institute of Technology}{}{}{}%Double major. GPA: 4.1}
% {The computer science degree involves courses in systems, algorithms, functional programming, and complexity theory. I supplemented these courses with electives in computer graphics and advanced algorithms.
% The math degree involves courses in abstract algebra, differential geometry, and analysis. I supplemented these courses with electives in algebraic topology and Riemannian geometry.}  % arguments 3 to 6 can be left empty


% PUBLICATIONS ----------------------------------

\section{Journal Articles}
\begin{etaremune}[itemsep=.25\baselineskip, leftmargin=2.75em, labelsep=1em]
\item \cvcite{Gillespie:2024:RTH}
\item \cvcite{Hirose:2024:SK}
\item \cvcite{Feng:2023:WND}
\item \cvcite{Liu:2023:SSI}
\item \cvcite{Gillespie:2021:ICI}
\item \cvcite{Gillespie:2021:DCE}
\end{etaremune}

% \section{Other Refereed Publications}
\section{Other Publications}
\begin{etaremune}[itemsep=.25\baselineskip, leftmargin=2.75em, labelsep=1em]
\item \cvcite{Gillespie:2024:EIT}
\item \cvcite{Hirose:2024:SKA}
\item \cvcite{Sharp:2021:GPI}
\end{etaremune}

\restoregeometry

%----------------------------------
\section{Other Research Experience}

% \cventry{2024--present}{Research Assistant}{Carnegie Mellon University}{}
% {Advisor: Keenan Crane}{}

% \cventry{2018--2024}{Graduate Researcher}{Carnegie Mellon University}{}
% {Advisor: Keenan Crane}{}

\cventry{July 2023}{Visiting Researcher}{Technische Universit\"at Berlin, Berlin}{}
        {Host: Boris Springborn}{}

\cventry{Summer 2022}{Visiting Graduate}{University of California, San Diego}{}
        {Host: Albert Chern}{}

% \cventry{Fall 2017}{Teaching Assistant for CS 171, Introduction to Computer Graphics}{California Institute of Technology}{}{}{Under Professor Alan Barr, graded problem sets, held weekly office hours, delivered recitation lectures}

\cventry{Summer 2017}{Arthur R. Adams Undergraduate Researcher}{Caltech}{}
{Mentor: Peter Schr\"oder}
{} % Implemented an energy-preserving integrator for 2D MHD on grids and proved its conservation properties
% \begin{itemize}%
% \item Analyzed conservation behavior of the algorithm using discrete differential geometry
% \item Implemented algorithm in Houdini
% \end{itemize}

\cventry{Summer 2016}{Arthur R. Adams Undergraduate Researcher}{Caltech}{}
{Mentor: Mathieu Desbrun}
{} % Developed an algorithm for computing polymer conformation using dimensionality reduction techniques.
% \begin{itemize}%
% \item Implemented algorithm in C++
% \item Experimented with applying the algorithm to point cloud denoising
% \end{itemize}

\cventry{2016--2017}{Undergraduate Researcher}{Caltech}{}
{Mentor: Alan Barr}
{}% Explored applications of interval analysis to root-finding and solving differential equations
% \begin{itemize}%
% \item Implemented interval analysis library in Haskell
% \item Implemented graphical viewer for interval root-finding and minimization algorithms
% \end{itemize}

% \cventry{Spring 2017, Spring 2016}{Teaching Assistant for CS 38, Introduction to Algorithms}{California Institute of Technology}{}{}{Under Professor Leonard Schulman, graded problem sets and held weekly office hours}

% \cventry{Summer 2015}{Software Engineering Intern}{Google}{}{}{}
% Prototyped new credit card entry interface for Android library. Developed in Java}



\section{Awards \& Honors}
\cvitemwithcomment{2024}{Two SIGGRAPH Best Paper Award Honorable Mentions}{Awarded to 12 papers out of $\sim$840 submissions; $\sim$top 1.5\% of papers}
\cvitemwithcomment{2019-2022}{NSF Graduate Research Fellowship}{Awarded to top 15\% of applicants across all areas of science; \$147,000 over 3 years}
\cvitemwithcomment{2016-2017}{Arthur R Adams SURF Fellow}{}
\cvitemwithcomment{2017}{SIGGRAPH ACM Turing Award Celebration Grant}{}
% \cvitemwithcomment{2016}{William Lowell Putnam Mathematics Competition }{31 points (rank: 365/3214)}

%----------------------------------
\section{Selected Talks}
\cventry
    {Nov. 2024}
    {Solid Knitting \& Harmonic Hitting}
    {IST Austria}{}{}
    {}
\cventry
    {Aug. 2024}
    {Ray Tracing Harmonic Functions}
    {ACM SIGGRAPH 2024}{}{}
    {}
\cventry
    {Sept. 2023}
    {Intrinsic Triangulations in Geometry Processing}
    {IST Austria}{}{}
    {}
\cventry
    {Aug. 2023}
    {Intrinsic Triangulations in Geometry Processing}
    {Geometry Workshop in Obergurgl}{}{}
    {}
\cventry
    {Jul. 2023}
    {Intrinsic Triangulations in Geometry Processing}
    {TU Berlin SFB TRR 109 Colloquium}{}{}
    {}
\cventry
    {Apr. 2022}
    {Discrete Conformal Equivalence of Polyhedral Surfaces}
    {UCSD Pixel Cafe}{}{}
    {}
\cventry
    {Nov. 2021}
    {Integer Coordinates for Intrinsic Geometry Processing}
    {ACM SIGGRAPH Asia 2021}{}{}
    {}
\cventry
    {Aug. 2021}
    {Discrete Conformal Equivalence of Polyhedral Surfaces}
    {ACM SIGGRAPH 2021}{}{}
    {}
\cventry
    {Aug. 2021}
    {Geometry Processing with Intrinsic Triangulations}
    {ACM SIGGRAPH 2021 Courses}{}{}
    {}
\cventry
    {June 2021}
    {Geometry Processing with Intrinsic Triangulations}
    {SIAM International Meshing Roundtable Courses (IMR 2021)}{}{}
    {}


% \cventry
%   {Oct. 2017}
%   {2D Plasma Simulation via Discrete Exterior Calculus}
%   {California Institute of Technology Summer Research Seminar Day}{}{}
%   {15 minute presentation on the results of my summer research}

% \cventry
%   {Sept. 2017}
%   {Combinatorics and the Probabilistic Method}
%   {Westfield High School Seminar in College Mathematics}{}{}
%   {30 minute presentation to a high school math class. Gave an introduction to elementary combinatorics and presented some simple applications of the probabilistic method}

% \cventry
%   {Mar. 2017}
%   {Continuous and Discrete Mechanics for Variational Integrators}
%   {California Institute of Technology CS 177b}{}{}
%   {1.5 hour final presentation for a computer graphics class. Gave an overview of Hamiltonian/Lagrangian mechanics and how to discretize them to produce variational time integrators}

% \cventry
%   {Dec. 2016}
%   {Measurement in Quantum Mechanics}
%   {Westfield High School Seminar in College Mathematics}{}{}
%   {30 minute presentation to a high school math class. Gave an introduction to projective measurements in Quantum Mechanics, working through the example of the Stern-Gerlach device}

% \cventry
%   {Oct. 2016}
%   {Computing Chromosome Embedding from Contact Frequencies}
%   {California Institute of Technology Summer Research Seminar Day}{}{}
%   {15 minute presentation on the results of my summer research}

%----------------------------------
% \section{Selected Classes Taken}

% \cvitemwithcomment
%   {CS 177ab}
%   {Discrete Differential Geometry}
%   {discrete study of: differential forms, deRham cohomology, Poisson problems, variational mechanics}

% \cvitemwithcomment
%   {CS 176}
%   {Introduction to Computer Graphics Research}
%   {geometry processing, data visualization, vector fields and flows}

% \cvitemwithcomment
%   {CS 171}
%   {Introduction to Computer Graphics Laboratory}
%   {shaders, geometry processing, physical simulation, ray tracing}

% \cvitemwithcomment
%   {Ma 151abc}
%   {Algebraic Topology}
%   {fundamental groups, covering spaces, homology, cohomology, spectral sequences, characteristic classes}

% \cvitemwithcomment
%   {Ma 157ab}
%   {Riemannian Geometry}
%   {Riemannian metrics, geodesics, vector fields, Riemannian curvature, Representation theory of Lie groups}

% \cvitemwithcomment
%   {CS 150}
%   {Probability and Algorithms}
%   {analysis of probabilistic algorithms, the probabilistic method}

% \cvitemwithcomment
%   {CS 139}
%   {Analysis and Design of Algorithms}
%   {streaming algorithms, experts algorithm, SDPs, spectral graph theory}


% \cvitemwithcomment
%   {PS 172}
%   {Bayesian Statistics}
%   {Bayesian updates, Markov Chain Monte Carlo}

%----------------------------------
%\section{Publications}
%\cventry{ongoing}{Smooth Embeddings from Pairwise Distances}{}{}{}
%{I am currently working with Professor Mathieu Desbrun to write up the work we did together to submit for publication}

  \section{Service}
  \cvitem{Departmental}{
          Organizer, Graphics Reading Group (2022-2023);
          Organizer, Graphics Seminar (2020-2021);
          Panel Speaker (CSD Visit Day 2020, 2023, CSD Introductory Course 2022); Organizer, PhD mutual mentorship pod (2022-2024)}
  \cvitem{Reviewing}{SIGGRAPH (2019, 2022, 2023, 2024), SIGGRAPH Asia (2022, 2023, 2024), ACM Transactions on Graphics (2024), Eurographics (2024), Computer Graphics Forum (2024), Journal of Computational and Applied Mathematics (2024), Computer-Aided Design (2023), Transactions on Visualization and Computer Graphics (2023, 2024), Computers \& Graphics (2021)}
  \cvitem{Mentorship}{Summer Geometry Initiative volunteer (2024), Advising Master's student (2022-2023), CMU Summer Undergraduate Research Fellowship (2020)}

\section{Press Coverage}

\cventry{July 2024}{Cosmos magazine}{
\href{https://cosmosmagazine.com/technology/computing/3d-knitting-could-make-solid-but-soft-furniture/}{3D knitting could make solid but soft furniture}}{}{}{}


\cventry{July 2024}{Interesting Engineering}{\href{https://interestingengineering.com/innovation/solid-knitted-three-dimensional-furniture-could-be-reality}{Beware IKEA: Solid knitted three-dimensional furniture could be a reality}}{}{}{}


\cventry{July 2024}{New Atlas}{\href{https://newatlas.com/materials/solid-knitting-reconfigurable/}{Innovative `solid knitting' machine builds 100\% reconfigurable objects}}{}{}{}


\cventry{July 2024}{ZME Science}{\href{https://www.msn.com/en-us/news/technology/solid-knitting-a-different-spin-on-3d-printing-that-can-make-furniture-out-of-yarn/ar-AA1ogaiP?ocid=BingNewsVerp}{Solid knitting: a different spin on 3D printing that can make furniture out of yarn}}{}{}{}


\cventry{July 2024}{ACM SIGGRAPH Blog}{\href{https://blog.siggraph.org/2024/07/beyond-the-threads.html/}{Beyond the Threads}}{}{}{}


\cventry{July 2024}{CMU News}{\href{https://www.cmu.edu/news/stories/archives/2024/july/robotics-institute-introduces-solid-knitting-as-new-fabrication-technique}{Robotics Institute Introduces Solid Knitting as New Fabrication Technique}}{}{}{}


  % ----------------------------------
  % \section{Programming Languages}
  % \cvitem{}{C++, Mathematica, Python, Java, Matlab, Haskell, Ocaml, \LaTeX}

\end{document}
